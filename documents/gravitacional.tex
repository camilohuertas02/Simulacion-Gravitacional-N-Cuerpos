\documentclass{article}
\usepackage[utf8]{inputenc}
\usepackage{amsmath}
\usepackage{geometry}
\geometry{a4paper, margin=1in}

\title{Simulación del Problema Gravitacional de N Cuerpos}
\author{Jules (Asistente AI)}
\date{\today}

\begin{document}
\maketitle

\section{Introducción}
Este documento describe la base teórica y la implementación de una simulación del movimiento de N cuerpos que interactúan mutuamente a través de la ley de la gravitación universal de Newton. El objetivo es simular este sistema sin realizar aproximaciones comunes, como suponer que una masa es dominante sobre las otras.

\section{Fundamento Físico}
Cada cuerpo de masa $m_i$, con un vector de posición $\vec{r}_i$, está sometido a la fuerza gravitacional ejercida por todos los demás cuerpos $m_j$ en el sistema. La segunda ley de Newton para el cuerpo $i$ se expresa como:
\begin{equation}
m_i \frac{d^2\vec{r}_i}{dt^2} = \vec{F}_i
\end{equation}
donde $\vec{F}_i$ es la fuerza neta sobre el cuerpo $i$. Esta fuerza es la suma vectorial de las fuerzas gravitacionales individuales ejercidas por cada uno de los otros $N-1$ cuerpos:
\begin{equation}
\vec{F}_i = \sum_{j \neq i}^{N} \vec{F}_{ij}
\end{equation}
La fuerza gravitacional $\vec{F}_{ij}$ ejercida por el cuerpo $j$ sobre el cuerpo $i$ está dada por la ley de la gravitación universal de Newton:
\begin{equation}
\vec{F}_{ij} = G \frac{m_i m_j}{|\vec{r}_j - \vec{r}_i|^3} (\vec{r}_j - \vec{r}_i)
\label{eq:fuerza_grav}
\end{equation}
donde $G$ es la constante gravitacional universal. El vector $(\vec{r}_j - \vec{r}_i)$ apunta desde el cuerpo $i$ hacia el cuerpo $j$, indicando que la fuerza es atractiva. La magnitud de la fuerza es inversamente proporcional al cuadrado de la distancia entre los cuerpos, $|\vec{r}_j - \vec{r}_i|^2$.

\section{Unidades y Constantes}
Para simplificar los cálculos numéricos, se utiliza un sistema de unidades normalizadas. En este sistema, la constante gravitacional $G$ se establece en 1:
\begin{equation}
G = 1.0
\end{equation}
Esto permite trabajar con masas, distancias y tiempos en términos relativos, sin necesidad de manejar los órdenes de magnitud extremadamente grandes o pequeños que aparecen al usar unidades estándar del SI.

\section{Integración Numérica: Algoritmo de Verlet}
Para resolver el sistema de ecuaciones diferenciales de segundo orden (Ecuación \ref{eq:fuerza_grav} combinada con la segunda ley de Newton), se emplea un método de integración numérica. El algoritmo de Verlet es una elección popular para simulaciones de dinámica molecular y sistemas N-cuerpos debido a su buena estabilidad a largo plazo y conservación de energía, además de ser reversible en el tiempo.

La forma del algoritmo de Verlet utilizada es la conocida como "velocity Verlet":
\begin{align}
\vec{r}_i(t + \Delta t) &= \vec{r}_i(t) + \vec{v}_i(t) \Delta t + \frac{1}{2} \vec{a}_i(t) (\Delta t)^2 \label{eq:verlet_r} \\
\vec{v}_i(t + \Delta t) &= \vec{v}_i(t) + \frac{1}{2} \left( \vec{a}_i(t) + \vec{a}_i(t + \Delta t) \right) \Delta t \label{eq:verlet_v}
\end{align}
donde:
\begin{itemize}
    \item $\vec{r}_i(t)$, $\vec{v}_i(t)$, y $\vec{a}_i(t)$ son la posición, velocidad y aceleración del cuerpo $i$ en el tiempo $t$, respectivamente.
    \item $\Delta t$ es el paso de tiempo de la integración.
    \item $\vec{a}_i(t) = \vec{F}_i(t) / m_i$.
    \item $\vec{a}_i(t + \Delta t)$ es la aceleración del cuerpo $i$ en el tiempo $t + \Delta t$, calculada usando las nuevas posiciones $\vec{r}_k(t + \Delta t)$ de todos los cuerpos $k$ obtenidas de la Ecuación \ref{eq:verlet_r}.
\end{itemize}
El procedimiento en cada paso de tiempo $\Delta t$ es:
\begin{enumerate}
    \item Calcular $\vec{F}_i(t)$ para cada cuerpo $i$ basándose en las posiciones $\vec{r}_k(t)$.
    \item Calcular $\vec{a}_i(t) = \vec{F}_i(t) / m_i$.
    \item Actualizar las posiciones a $\vec{r}_i(t + \Delta t)$ usando la Ecuación \ref{eq:verlet_r}.
    \item Calcular las nuevas fuerzas $\vec{F}_i(t + \Delta t)$ usando las nuevas posiciones $\vec{r}_k(t + \Delta t)$.
    \item Calcular las nuevas aceleraciones $\vec{a}_i(t + \Delta t) = \vec{F}_i(t + \Delta t) / m_i$.
    \item Actualizar las velocidades a $\vec{v}_i(t + \Delta t)$ usando la Ecuación \ref{eq:verlet_v}.
\end{enumerate}

\section{Conservación de la Energía}
En un sistema gravitacional aislado, la energía mecánica total $E$ debe conservarse. Esta energía es la suma de la energía cinética total $K$ y la energía potencial total $U$:
\begin{equation}
E = K + U
\end{equation}
La energía cinética total es:
\begin{equation}
K = \sum_{i=1}^{N} \frac{1}{2} m_i |\vec{v}_i|^2
\end{equation}
La energía potencial gravitacional total es la suma de las energías potenciales de todos los pares de cuerpos únicos:
\begin{equation}
U = -\sum_{i=1}^{N} \sum_{j=i+1}^{N} G \frac{m_i m_j}{|\vec{r}_i - \vec{r}_j|}
\end{equation}
El seguimiento de la energía total del sistema a lo largo de la simulación es una forma importante de verificar la precisión y estabilidad del integrador numérico.

\section{Implementación del Software}
El software se desarrolla en C++ utilizando un enfoque de Programación Orientada a Objetos (POO).
\begin{itemize}
    \item \textbf{Clase `vector3D`}: Proporciona funcionalidades para álgebra vectorial en 3 dimensiones (suma, resta, producto escalar, producto vectorial, norma, etc.). Esta clase es fundamental para manejar las cantidades vectoriales como posición, velocidad y fuerza.
    \item \textbf{Clase `Cuerpo`}: Encapsula las propiedades de cada cuerpo celeste (masa $m$, radio $R$, posición $\vec{r}$, velocidad $\vec{V}$, y fuerza $\vec{F}$). Incluye métodos para inicializar el cuerpo, calcular la fuerza gravitacional debida a otro cuerpo, y actualizar su estado (posición y velocidad) mediante el algoritmo de Verlet.
    \item \textbf{Lógica Principal (`main.cpp`)}: Gestiona la simulación completa, incluyendo:
    \begin{itemize}
        \item Solicitud de datos iniciales al usuario (número de cuerpos, sus propiedades, parámetros de simulación como $\Delta t$ y tiempo total).
        \item Verificación de la validez de los datos de entrada.
        \item El bucle principal de simulación que itera en el tiempo, aplicando el algoritmo de Verlet.
        \item Cálculo y almacenamiento de las fuerzas, posiciones, velocidades y energías en cada paso.
        \item Salida de resultados a un archivo de datos.
        \item Interfaz para la visualización de resultados mediante herramientas externas (Gnuplot, Python/Matplotlib, Octave).
    \end{itemize}
\end{itemize}
La estructura del proyecto está organizada en directorios para `include` (archivos de cabecera), `src` (archivos fuente), `bin` (ejecutable), `results` (datos de salida), `scripts` (scripts de graficación) y `documents` (documentación). Un `Makefile` gestiona la compilación del proyecto, la generación de documentación (Doxygen) y la creación de este documento PDF (LaTeX).

\end{document}
